

\documentclass[conference]{IEEEtran}

\usepackage{adjustbox}




% correct bad hyphenation here
\hyphenation{op-tical net-works semi-conduc-tor}


\begin{document}

\title{Assignment Helper\\ Empowering Your Programming Skill \\ Power OVERWHELMING co.}

% author names and affiliations
% use a multiple column layout for up to three different
% affiliations
% \author{\IEEEauthorblockN{Jae-gook Kim}
% \IEEEauthorblockA{Department of Information System\\
% Hanyang University, Seoul,  Korea\\
% Email: claretta@hanyang.ac.kr}
% \and
% \IEEEauthorblockN{Kyung-min Kim}
% \IEEEauthorblockA{Department of Information System\\
% Hanyang University, Seoul,  Korea\\
% Email: kimkmhk@hanyang.ac.kr}
% \IEEEauthorblockN{Ji-hoon Lee}
% \IEEEauthorblockA{Department of Finance,\\ Business School,\\
% Hanyang University, Seoul,  Korea\\
% Email: 1equal2@hanyang.ac.kr}
% \and
% \IEEEauthorblockN{Kyo-ho Lee}
% \IEEEauthorblockA{Department of Information System\\
% Hanyang University, Seoul,  Korea\\
% Email: 1equal2@hanyang.ac.kr}}


% conference papers do not typically use \thanks and this command
% is locked out in conference mode. If really needed, such as for
% the acknowledgment of grants, issue a \IEEEoverridecommandlockouts
% after \documentclass

% for over three affiliations, or if they all won't fit within the width
% of the page, use this alternative format:
% 
\author{\IEEEauthorblockN{Ji-hoon Lee\IEEEauthorrefmark{1},
Jae-gook Kim\IEEEauthorrefmark{2},
Kyung-min Kim\IEEEauthorrefmark{3}, and
Kyo-ho Lee\IEEEauthorrefmark{4} }
\IEEEauthorblockA{\IEEEauthorrefmark{2}\IEEEauthorrefmark{3}\IEEEauthorrefmark{4}Department of Information System}
\IEEEauthorblockA{\IEEEauthorrefmark{1}Department of Finance, Business School}
\IEEEauthorblockA{Hanyang University, Seoul, Korea}
\IEEEauthorblockA{Email: \IEEEauthorrefmark{1}starypoc@hanyang.ac.kr,  \IEEEauthorrefmark{2}claretta@hanyang.ac.kr, \IEEEauthorrefmark{3}kimkmhk@hanyang.ac.kr, \IEEEauthorrefmark{4}1equal2@hanyang.ac.kr}}



\maketitle

\begin{abstract}
Assignment Helper is a program to help people who have difficulties with their programming language assignments. Assignment Helper will able to help the assignment by crawling the data from webs such as programmer forums and then find similar questions and related codes. It will also have a verification function and will check the codes by using CRC or other compilers. It shall check it by testing the input, output and the expectation all together.

\centering


\begin{table}[h]
\renewcommand{\arraystretch}{1.3}
\caption{Role Assignment}
\label{table_role}
\centering
\begin{adjustbox}{width=0.5\textwidth}
\small
\begin{tabular}{c||c||c}
\hline
\bfseries Role & \bfseries Name & \bfseries Task Description \\
\hline\hline
Developer Manager  & Kim Jae-gook & \parbox[t]{5cm}{Managing whole process of developing\\ the program. }\\
\hline
Users & Lee Ji-hoon & \parbox[t]{5cm}{Seeking for the usefulness compare\\ to other similar programs.}\\
\hline
Cutomers & Kim Kyung-min & \parbox[t]{5cm}{Finding out whether the project is \\ valuable.}\\
\hline
Developer & Lee Kyu-ho & \parbox[t]{5cm}{Implementing the program.}\\
\hline

\end{tabular}
\end{adjustbox}

\end{table}
\end{abstract}

\IEEEpeerreviewmaketitle


\section{Introduction}

\subsection{Motivation}
We have started this project with the motivation to help students (especially freshman) who are struggling with their assignments. Other assignments like history or natural science can easily find their assignment sources quickly from just googling it simply. However, it is impossible to do that simple procedure in programming language assignment. Without perfect knowledge about the computer languages, it comes out difficult to solve their programming language assignment. 

We discussed about a solution to solve this. We concluded that it is not easy to study programming languages especially to the students who studied only at school and have never lived closed to computer languages; The students who only studied subjects for KSAT, the entrance exam for university!

\subsubsection{Subsubsection Heading Here}
Subsubsection text here.


\begin{thebibliography}{1}

\bibitem{IEEEhowto:kopka}
H.~Kopka and P.~W. Daly, \emph{A Guide to \LaTeX}, 3rd~ed.\hskip 1em plus
  0.5em minus 0.4em\relax Harlow, England: Addison-Wesley, 1999.

\end{thebibliography}




% that's all folks
\end{document}


