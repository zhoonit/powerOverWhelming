

\documentclass[conference]{IEEEtran}

\usepackage{adjustbox}
\usepackage{graphicx}
\usepackage{alphalph}
\usepackage{url}
\usepackage[colorlinks]{hyperref}

\let\OLDitemize\itemize
\renewcommand\itemize{\OLDitemize\addtolength{\itemsep}{1em}}
\let\OLDenumerate\enumerate
\renewcommand\enumerate{\OLDenumerate\addtolength{\itemsep}{1em}}
\renewcommand\thesubsectiondis{\AlphAlph{\value{subsection}}.}% for the headings in the text
\renewcommand\thesubsection{\mbox{\thesection-\AlphAlph{\value{subsection}}}}% for the ToC, for example     
\newcommand{\cmt}[1]{\PackageWarning{comment}{#1}}

% correct bad hyphenation here
\hyphenation{op-tical net-works semi-conduc-tor}


\begin{document}

\title{Coding Helper\\ Empowering Your Programming Skill \\ Power OVERWHELMING co.}

% author names and affiliations
% use a multiple column layout for up to three different
% affiliations
% \author{\IEEEauthorblockN{Jae-gook Kim}
% \IEEEauthorblockA{Department of Information System\\
% Hanyang University, Seoul,  Korea\\
% Email: claretta@hanyang.ac.kr}
% \and
% \IEEEauthorblockN{Kyung-min Kim}
% \IEEEauthorblockA{Department of Information System\\
% Hanyang University, Seoul,  Korea\\
% Email: kimkmhk@hanyang.ac.kr}
% \IEEEauthorblockN{Ji-hoon Lee}
% \IEEEauthorblockA{Department of Finance,\\ Business School,\\
% Hanyang University, Seoul,  Korea\\
% Email: 1equal2@hanyang.ac.kr}
% \and
% \IEEEauthorblockN{Kyo-ho Lee}
% \IEEEauthorblockA{Department of Information System\\
% Hanyang University, Seoul,  Korea\\
% Email: 1equal2@hanyang.ac.kr}}


% conference papers do not typically use \thanks and this command
% is locked out in conference mode. If really needed, such as for
% the acknowledgment of grants, issue a \IEEEoverridecommandlockouts
% after \documentclass

% for over three affiliations, or if they all won't fit within the width
% of the page, use this alternative format:
% 
\author{\IEEEauthorblockN{Ji-hoon Lee\IEEEauthorrefmark{1},
Jae-gook Kim\IEEEauthorrefmark{2},
San Gu\IEEEauthorrefmark{3}, and
Kyoo-ho Lee\IEEEauthorrefmark{4} }
\IEEEauthorblockA{\IEEEauthorrefmark{2}\IEEEauthorrefmark{3}\IEEEauthorrefmark{4}Department of Information System}
\IEEEauthorblockA{\IEEEauthorrefmark{1}Department of Finance, Business School}
\IEEEauthorblockA{Hanyang University, Seoul, Korea}
\IEEEauthorblockA{Email: \IEEEauthorrefmark{1}starypoc@hanyang.ac.kr,  \IEEEauthorrefmark{2}claretta@hanyang.ac.kr, \IEEEauthorrefmark{3}hyunsichang@hanyang.ac.kr, \IEEEauthorrefmark{4}1equal2@hanyang.ac.kr}}

% \author{\IEEEauthorblockN{Ji-hoon Lee\IEEEauthorrefmark{1},
% Jae-gook Kim\IEEEauthorrefmark{2}, and
% San Gu\IEEEauthorrefmark{3} }
% \IEEEauthorblockA{\IEEEauthorrefmark{2}\IEEEauthorrefmark{3}\IEEEauthorrefmark{4}Department of Information System}
% \IEEEauthorblockA{\IEEEauthorrefmark{1}Department of Finance, Business School}
% \IEEEauthorblockA{Hanyang University, Seoul, Korea}
% \IEEEauthorblockA{Email: \IEEEauthorrefmark{1}starypoc@hanyang.ac.kr,  \IEEEauthorrefmark{2}claretta@hanyang.ac.kr, \IEEEauthorrefmark{3}hyunsichang@hanyang.ac.kr} }




\maketitle

\begin{abstract}
Coding Helper is a program to help people who have difficulties with their programming language assignments.  It is designed to be easy to use for those who are new to programming and are looking for examples easily. This can also be helpful to programmers who are somewhat skilled in finding examples when creating new programs. Coding Helper receives inputs in natural language for those who are unfamiliar with programming. The user then chooses which programming language he wants as the result. This then finds the keyword in that natural language and finds the programming data corresponding to that keyword. It will able to help the assignment by crawling the data from webs such as programmer forums like "github" and then find similar questions and related codes. It will also have a verification function and will check the codes by using CRC or other compilers. It shall check it by testing the input, output and the expectation all together.


\centering


\begin{table}[h]
\renewcommand{\arraystretch}{1.3}
\caption{Role Assignment}
\label{table_role}
\centering
\begin{adjustbox}{width=0.5\textwidth}
\small
\begin{tabular}{c||c||c}
\hline
\bfseries Role & \bfseries Name & \bfseries Task Description \\
\hline\hline
Developer Manager & Gu San & \parbox[t]{5cm}{Managing whole process of developing\\ the program. }\\
\hline
Users & Lee Ji-hoon & \parbox[t]{5cm}{Seeking for the usefulness compare\\ to other similar programs.}\\
\hline
Cutomers & Kim Jae-kook & \parbox[t]{5cm}{Finding out whether the project is \\ valuable.}\\
\hline
Developer & Lee Kyoo-ho & \parbox[t]{5cm}{Implementing the program.}\\
\hline


\end{tabular}
\end{adjustbox}

\end{table}
\end{abstract}

\IEEEpeerreviewmaketitle


\section{Introduction}

\subsection{Motivation}
We have started this project with the motivation to help students (especially freshman) who are struggling with their assignments. Other assignments like history or natural science can easily find their coding sources quickly from just googling it simply. However, it is impossible to do that simple procedure in programming language assignment. Without perfect knowledge about the computer languages, it comes out difficult to solve their programming language assignment. 

We discussed about a solution to solve this. We concluded that it is not easy to study programming languages especially to the students who studied only at school and have never lived closed to computer languages; The students who only studied subjects for KSAT, the entrance exam for university!

\subsection{Problem Statement}
It is problematic for those who first met the programming language and getting no additional help. It can be not only unfair but also can cause inefficiency in terms of the social resource allocation. Maybe, some novice students might have natural talents at computer programming. But others needs opportunities to catch up. So, we are going to build a program to ease the life of these novice students.


\subsection{Research on Any Related Software}
Our program aims for helping students who is unfamiliar with programming paradigms. There are some services which provide help to those students in primitive levels than our service.

\subsubsection{Education Service: Scratch}
There are many programs for teaching programing language, but Scratch is the most representative one. Scratch does not execute commands by entering code with long instructions. Instead, click and drag the block to move the feature to perform the command. Scratch is a popular child-coding tool for schools and institutions worldwide. Scratch's intuitive interface is useful but has difficulty to find right answer. If one wants to make a program working properly. He should keep try until it matches. It is very frustrating for those who want to know the correct answer right away.

\subsubsection{Easier Coding: Emmet}
Emmet is a programming language that makes it easy to write CSS, even though it can save your coding time, the language you should learn for one programming language increases. Also, if you are not aware of the language, it is hard to get used to it. Instead, Coding Helper will provide the coding style with natural language processing.

\subsubsection{Crowdsourcing and Knowledge Sharing: Stackoverflow}
Stackoverflow is a site where programmers ask and answers about programming. Stackoverflow is the largest developer community on the scale. It will be a good idea to ask questions here for the answers comes up very quickly.
Since there are a lot of questions that has been answered, the problems that you need answers are mostly up on the forum already. In other words, rather than asking, you are more likely to search and get answers. But the point is for the beginners, it can even be hard to determine what to find. You must define the search keywords yourself. It would be a big hurdle for beginners. The limitation is that there is no recommend keywords.

\subsection{Distinguishable Features of Coding Helper} % (fold)
\label{sub:distinguishable_features_of_assignment_helper}
There are countless amounts of services that tries to help the problem we defined. Our Coding Helper will have differentiating features from existing services. Some distinguishable features are as follows:

\subsubsection{Lightening The Burden On Determining Search Keywords}
For some questions for beginners, it could be hard to determine a search keyword by themselves. For example, if they want to know what list object can `do', it is appropriate to google on `list method in python'. To google the word `method', the user should know the word beforehand, which is practically impossible without any additional help.


Coding Helper will allow to search on more natural-language-like queries, making the user easy to search with their actual thoughts.

\subsubsection{Grouping Related Questions}
With natural-language-like queries, it could be hard to guess what the user intended from the very beginning. Coding Helper will explore broad range of possibilities and suggest each to the users.

\subsubsection{Getting the Working Codes Right Away from the Internet}
There are numerous codes that are floating on the internet. But it is not verified whether the codes on the internet works properly or not. Coding Helper will check the validity of the code automatically 

% subsection distinguishable_features_of_assignment_helper (end)

\section{Requirements} % (fold)
\label{sec:requirements}


\textit{\textbf{Frontend Related}}
\textit{ }

\subsection{Building installation Manual}
 The program is for beginner, so it provide easy installation manual.

\begin{itemize}
  \item manual file
  \begin{itemize}
    \item The user will be able to understand the installation intuitively.
  \end{itemize}
  \item One zip file construction
  \begin{itemize}
    \item Because the file is provided as a single file on the Internet, the only thing the user need to do is download and unzip the file.
  \end{itemize}
\end{itemize}
\textit{ }


\subsection{Building Portable Client}
 Portable client can help the user easy to use the program regardless whichever OS user uses

\begin{itemize}
  \item Building one window composition
  \begin{itemize}
    \item This will help user understand how program works easily
  \end{itemize}
\end{itemize}
\textit{ }


\subsection{Building instruction for Users}
 Instructions help user run the program are provided by text file, brief movie, and tooltips

\begin{itemize}
  \item Building Readme.txt
  \begin{itemize}
    \item Give basic instruction
  \end{itemize}
  \item Building basic tutorial.mov
  \begin{itemize}
    \item Give demo-like tutorial
  \end{itemize}
  \item Building tooltip
  \begin{itemize}
    \item When hovering on a button for 3 seconds, program show up a tool tip
  \end{itemize}
\end{itemize}
\textit{ }


\subsection{Executing Client}
 Executing the main program 

\begin{itemize}
  \item Opening the main program
  \item Program shows up within 1 minute
\end{itemize}
\textit{ }


\subsection{Closing Client}
 Closing Whole Program 

\begin{itemize}
  \item Program should be terminated whenever user wants
  \item Whole process should be terminated within 1 minute, without error
\end{itemize}
\textit{ }


\subsection{Building Select UI}
 User interface for search with ease to use

\begin{itemize}
  \item search window
  \begin{itemize}
    \item A window from which users can start searching
  \end{itemize}
  \item query textbox
  \begin{itemize}
    \item A textbox from which users can make a query in text type
  \end{itemize}
  \item select code checkbox
  \begin{itemize}
    \item A checkbox from which users can decide what program language to show result
  \end{itemize}
  \item Expandability   \begin{itemize}
    \item Can scroll infinitely (like Facebook) in case there are lot of results
  \end{itemize}
  \item search button   \begin{itemize}
    \item A button from which users can start the search
  \end{itemize}
\end{itemize}
\textit{ }


\subsection{Building Candidate UI}
 User interface for select code with ease to use

\begin{itemize}
  \item Candidate window
  \begin{itemize}
    \item A window from which users can select the code 
  \end{itemize}
  \item query textbox
  \begin{itemize}
    \item A textbox from which users can see the code in several program language
  \end{itemize}
  \item select code checkbox
  \begin{itemize}
    \item A checkbox from which users can select which program language to show result
  \end{itemize}
  \item Expandability 
  \begin{itemize}
    \item Can scroll infinitely in case there are lot of results
  \end{itemize}
  \item Compile button 
  \begin{itemize}
    \item A button from which users can start compile
  \end{itemize}
\end{itemize}
\textit{ }


\subsection{Building compile UI}
 User interface for showing progress of the work with ease to use

\begin{itemize} 
  \item compile window
  \begin{itemize}
    \item A window showing progress of compile
  \end{itemize}
  \item code showing textbox
  \begin{itemize}
    \item A textbox from which users can see the code in selected program language
  \end{itemize}
  \item progress bar
  \begin{itemize}
    \item A Bar showing the progress of the task as a percentage
  \end{itemize}
  \item Expandability   
  \begin{itemize}
    \item Can scroll infinitely in case there are lot of results
  \end{itemize}
  \item abort button
  \begin{itemize}
    \item A button from which users can abort compile
  \end{itemize}
\end{itemize}
\textit{ }


\subsection{Building compile\_success UI}
 User interface for showing successful compiled code with ease to copy, apply or expand

\begin{itemize}
  \item success window
  \begin{itemize}
    \item A window showing the successful compiled result of selected code
  \end{itemize}
  \item code showing textbox
  \begin{itemize}
    \item A textbox from which users can see the code in selected program language
  \end{itemize}
  \item inputbox
  \begin{itemize}
    \item A textbox from which users can test input
  \end{itemize}
  \item outputbox
  \begin{itemize}
    \item A textbox from which users can test input
  \end{itemize}
  \item copy button
  \begin{itemize}
    \item button to copy code snippets to clipboard
  \end{itemize}
\end{itemize}
\textit{ }


\subsection{Building compile\_fail UI}
 User interface for select code with ease to use

\begin{itemize}
  \item fail window
  \begin{itemize}
    \item A window showing the failed result of selected code
  \end{itemize}
  \item code showing textbox
  \begin{itemize}
    \item A textbox from which users can see the code in selected program language
  \end{itemize}
  \item error showing textbox
  \begin{itemize}
    \item A textbox from which users can see the error code
  \end{itemize}
  \item copy button
  \begin{itemize}
    \item A button from which users can copy the code in selected program language
  \end{itemize}
\end{itemize}
\textit{ }


\textit{\textbf{Backend Related}}
\textit{ }

\subsection{Search Query Processing}
\begin{itemize}
  \item Natural Language Processing Technique
  \begin{itemize}
    \item Understanding the input with NLP
  \end{itemize}
  \item DB for search keyword
  \begin{itemize}
    \item Collecting keywords for seraching codes in DB to maximize accuracies
  \end{itemize}
  \item Function to extract keyword from input comparing to DB for search keyword
\end{itemize}
\textit{ }


\subsection{Learning System for Maximizing Accuracy}
\begin{itemize}
  \item Collect data from user and get feedback from them
\end{itemize}
\textit{ }


\subsection{Crawling}
\begin{itemize}
  \item First search
  \begin{itemize}
    \item Fast search from programmer forum
    \item Crawling data from Stackoverflow
  \end{itemize}
  \item Second search
  \begin{itemize}
    \item Broad search through search engine
    \item Crawling data from Google
  \end{itemize}
\end{itemize}
\textit{ }


\subsection{Parsing}
\begin{itemize}
  \item Parsing the codes using tree structure
  \begin{itemize}
    \item Check tree structure whether it is code part or not
  \end{itemize}
\end{itemize}
\textit{ }


\subsection{Grammar Check}
\begin{itemize}
  \item Save code into code file
  \begin{itemize}
    \item Before compiling code, convert string into compilable file
  \end{itemize}
  \item Conduct compiling code test
  \begin{itemize}
    \item Compiling code with GNU Compiler Collection
  \end{itemize}
\end{itemize}
\textit{ }


\subsection{Copy Code}
  \begin{itemize}
    \item Copy whole found string into clipboard
  \end{itemize}
\textit{ }


\subsection{Error Detection}
\begin{itemize}
  \item Detect error code
  \begin{itemize}
    \item When compile test is failed, find which error is occur
  \end{itemize}
  \item Print error string
  \begin{itemize}
    \item Translate error code into string for user to understand
  \end{itemize}
\end{itemize}
\textit{ }


\subsection{Edit Code Snippet}
  \begin{itemize}
    \item enable user to edit code manually when the compile test ended with error
  \end{itemize}
\textit{ }


\section{DEVELOPMENT ENVIRONMENT} % (fold)
\label{sec:development_environment}
\subsection{Choice of Software Development Platform} % (fold)
\label{sub:choice_of_software_development_platform}

\begin{enumerate}
  \item Operating System
  \begin{itemize}
    \item OS: Windows
    \item Reason: The program aims for students who are not familiar with programming. They are highly likely to use Windows and have few knowledge on other platform such as Linux.
  \end{itemize}
  \item Language and Platform
  \begin{itemize}
    \item Language and Platform: Python 3.6 with PyQt5
    \item Reason: Easy to implement crawling and parsing dealing with text segments as well as leaving rooms to use other packages.
  \end{itemize}
\end{enumerate}
\textit{}

% subsection choice_of_software_development_platform (end)

\subsection{Software in Use} % (fold)
\label{sub:software_in_use}

\begin{enumerate}
  \item \textit{Sublime Text 3}: Text editor for general use
  \item \textit{qt designer 5.90}: GUI based GUI designer for Qt environment
  \item \textit{pyCharm}: genral IDE for Python
  \item \textit{pandas}: Python package to deal with datas
  \item \textit{PyQt5}: Python package to deal with Qt gui
  \item \textit{stackexchage REST API}: REST API to do stack overflow search and get datas
  \item \textit{beautifulsoup4}: Python package for parsing XML and HTML
  \item \textit{requests}: Python package that assists to issue an HTML Request
  \item \textit{gcc}: C compiler to verify given source codes
\end{enumerate}
\textit{}
% subsection software_in_use (end)
% section development_environment (end)


\section{SPECIFICATIONS} % (fold)
\label{sec:specifications}

\begin{figure}[h]
\centering
\includegraphics[width=0.5\textwidth]{./figures/Process_Flow.png}
Our targeted users do not know how to code.

Coding Helper gives the code.
\caption{Process Flow of Coding Helper}
\label{fig_process_flow}
\end{figure}


\textbf{Getting Started}

\subsection{Setting - Easy Installation}

\begin{enumerate}
  \item Description
  \begin{itemize}
    \item Executable without installation to facilitate ease of use
    \item Works like portable utility
    \item Basic files only provide barebone program
    \begin{itemize}
      \item Programming language to search need be installed later
    \end{itemize}
  \end{itemize}
  \item Process(or I/O)
  \begin{enumerate}
    \item Download files from website
  \end{enumerate}
\end{enumerate}
\textit{}


\subsection{Setting - Configuration Window}

\begin{enumerate}
  \item Description
  \begin{itemize}
    \item Basic program is not able to use without installing programming language
    \item Provide check window
    \begin{itemize} 
      \item Check each window to install a language pack for a certain language 
    \end{itemize}
  \end{itemize}
  \item Process(or I/O)
  \begin{enumerate}
    \item Input: Opening conf.exe
    \item Output: Show conf.exe
  \end{enumerate}
\end{enumerate}

\textit{}

\subsection{Setting - Installing Language Pack}
\begin{enumerate}
  \item Description
  \begin{itemize}
    \item Basic program is not able to use without installing programming language
    \item Provide check window
    \begin{itemize}
      \item Check each window to install a language pack for a certain language 
    \end{itemize}
  \end{itemize}
  \item Process(or I/O)
  \begin{enumerate}
    \item Input: Selecting language packs and proceed 
    \item Output: Pack install on the hard disk
  \end{enumerate}
\end{enumerate}


\textit{}
\subsection{Instruction - README}
\begin{enumerate}
  \item Description
  \begin{itemize}
    \item Provide README file to give basic instruction
  \end{itemize}
  \item Process(or I/O)
  \begin{enumerate}
    \item N/A
  \end{enumerate}
\end{enumerate}


\textit{}

\subsection{Instruction - Basic Tutorial}
\begin{enumerate}
  \item Description
  \begin{itemize}
    \item Give demo-like tutorial on the very first run
  \end{itemize}
  \item Process(or I/O)
  \begin{enumerate}
    \item Input: Clicking tutorial on main window
    \item Input2: Initial program execution
    \item Output: Install selected pack on the hard disk
  \end{enumerate}
\end{enumerate}


\textit{}

\subsection{Instruction - On the Fly}
\begin{enumerate}
  \item Description
  \begin{itemize}
    \item Give tooltips on buttons
  \end{itemize}
  \item Process(or I/O)
  \begin{enumerate}
    \item Input: Hovering on a button for 3 seconds
    \item Output: Show up a tool tip
  \end{enumerate}
\end{enumerate}

\textit{}

\textbf{Execution}

\subsection{Execution}
\begin{enumerate}
  \item Description
  \begin{itemize}
    \item Executing the main program
  \end{itemize}
  \item Process(or I/O)
  \begin{enumerate}
    \item Opening the main program(helper.exe)
    \item Program(Process1. UI) shows up within 1 minute
  \end{enumerate}
\end{enumerate}

\textit{}

\subsection{Closing Whole Program}
\begin{enumerate}
  \item Description
  \begin{itemize}
    \item Closing Whole Program
  \end{itemize}
  \item Process(or I/O)
  \begin{enumerate}
    \item Clicking X window on search UI
    \item Program closing
  \end{enumerate}
\end{enumerate}
\textit{}


%프롬프트
\textbf{Process 1. Searching}


\textit{}
\begin{figure}[h]
\centering
\includegraphics[width=0.5\textwidth]{./figures/UI_main.jpg}
%text slot
\caption{Concept Image of Search Window}
\label{fig_concept_main}
\end{figure}


\subsection{Search - UI}
\begin{enumerate}
\item Description
\begin{itemize}
   \item Give instinctive search home
\end{itemize}
\item Process(or I/O)
  \begin{enumerate}
     \item Clicking search button on search UI
     \item Search begins
  \end{enumerate}
\end{enumerate}
\textit{}

\subsection{Search – History}
\begin{enumerate}
\item Description
\begin{itemize}
   \item Show searh history
\end{itemize}
\item Process(or I/O)
  \begin{enumerate}
     \item Output: Show up top 5 from search history list
  \end{enumerate}
\end{enumerate}
\textit{}

\subsection{Search – Option}
\begin{enumerate}
\item Description
\begin{itemize}
  \item Choose in what programming language result represent
  \begin{itemize}
     \item Show language option buttons
  \end{itemize}
\end{itemize}
\item Process(or I/O)
  \begin{enumerate}
    \item Input: click on a python button 
    \item Output: Show result on a python language 
    \item Input: click on a C button 
    \item Output: Show result on a C language 
    \item Input: click on a C++ button 
    \item Output: Show result on a C++ language 
  \end{enumerate}
\end{enumerate}
\textit{}

\subsection{Search – Auto Completion}
\begin{enumerate}
\item Description
\begin{itemize}
  \item Show match from search ketwords.csv
\end{itemize}
\item Process(or I/O)
  \begin{enumerate}
     \item Input: Each letters typed
    \item Output: Show all match words from search ketwords.csv
  \end{enumerate}
\end{enumerate}
\textit{}

\subsection{Request Submission by Key Press}
\begin{enumerate}
\item Description
\begin{itemize}
  \item Submit request by typing enter 
\end{itemize}
\item Process(or I/O)
  \begin{enumerate}
  \item Input: type enter key
  \item Output: Submit request
  \end{enumerate}
\end{enumerate}
\textit{}

\subsection{Request Submission by Clicking}
\begin{enumerate}
\item Description
\begin{itemize}
  \item Submit request by click 
\end{itemize}
\item Process(or I/O)
  \begin{enumerate}
     \item Input: mouse click
    \item Output: Submit request
  \end{enumerate}
\end{enumerate}
\textit{}

\subsection{Request Submission - Waiting UI}
\begin{enumerate}
\item Description
\begin{itemize}
  \item Halt user interface while submission 
\end{itemize}
\item Process(or I/O)
  \begin{enumerate}
     \item input: Submit request 
     \item Output: Freeze UI
  \end{enumerate}
\end{enumerate}
\textit{}

\subsection{Request Submission - Abort}
\begin{enumerate}
\item Description
   \begin{itemize}
  \item User press cancel
  \item Error occurs
  \item End submission
  \item Back to search window 
\end{itemize}
\item Process(or I/O)
  \begin{enumerate}
     \item input: Click cancel button
     \item input: Error occurs 
     \item Output: Stop submission
     \item Output: Back to start page
  \end{enumerate}
\end{enumerate}
\textit{}

\subsection{Request Submission – Extracting Keyword}
\begin{enumerate}
\item Description
\begin{itemize}
  \item Extract keywords from submitted  
\end{itemize}
\item Process(or I/O)
  \begin{enumerate}
     \item input: Submit request 
     \item Output: Freeze UI
  \end{enumerate}
\end{enumerate}
\textit{}

\subsection{Request Processing - Crawling(Stackoverflow)}
\begin{enumerate}
\item Description
\begin{itemize}
  \item Crawl popular codes
  \item Organize codes by keywords
  \item Process(or I/O) 
  \item Crawl Stackoverflow by keywords.csv
  \item Give code with popularity over 3/5 to code page
\end{itemize}
\begin{enumerate}
     \item Make keyword a key 
     \item Organize code page by keyword order
  \end{enumerate}
\end{enumerate}
\textit{}

\subsection{Request Processing - Crawling(Google)}
\begin{enumerate}
\item Description

\begin{itemize} 
  \item Crawl popular codes
  \item Organize codes by keywords
  \item Process(or I/O)
  \item Crawl Google by keywords.csv
  \item Give code with popularity over 3/5 to code page
\end{itemize}
  \begin{enumerate}
     \item Make keyword a key 
     \item Organize code page by keyword order
  \end{enumerate}
\end{enumerate}
\textit{}

\subsection{Request Processing - Detect Code Part}
\begin{enumerate}
\item Description
  \begin{itemize}
    \item Check tree structure whether it is code part or not
  \end{itemize}
  \item Process(or I/O)
  \begin{enumerate}
    \item Encode code page
    \item output: error code
  \end{enumerate}
\end{enumerate}
\textit{}


\subsection{Request Completed - UI}
\begin{enumerate}
  \item Description
  \begin{itemize}
    \item Give code page to selection
  \end{itemize}
  \item Process(or I/O)
  \begin{enumerate}
    \item Give all crawled code page to selection phase
  \end{enumerate}
\end{enumerate}

\textit{}

\textbf{Process 2. Code Selction}


\textit{}
\begin{figure}[h]
\centering
\includegraphics[width=0.5\textwidth]{./figures/UI_code_select.jpg}
%text slot
\caption{Concept Image of Code Select Window}
\label{fig_concept_code_select}
\end{figure}


\subsection{Code Selection - Basic UI}

\begin{enumerate}
  \item Description
  \begin{itemize}
    \item User should select code among results before program starts other process
  \end{itemize}
  \item Process(or I/O)
  \begin{enumerate}
    \item Clicking code among results
    \item Next process starts within 5 seconds
  \end{enumerate}
\end{enumerate}


\textit{}

\subsection{Auto Compile Test - Requesting}

\begin{enumerate}
  \item Description
  \begin{itemize}
    \item Right after user selects code, program is compiling the code(See Figure~\ref{fig_concept_code_compile} )
  \end{itemize}
  \item Process(or I/O)
  \begin{enumerate}
    \item Input : Text that user chose
    \item Output: Test. Program format
  \end{enumerate}
\end{enumerate}

\textit{}
\begin{figure}[h]
\centering
\includegraphics[width=0.5\textwidth]{./figures/UI_code_validation.jpg}
%text slot
\caption{Concept Image of Code Compiling}
\label{fig_concept_code_compile}
\end{figure}

\textit{}

\subsection{Auto Compile Test - Success}

\begin{enumerate}
  \item Description
  \begin{itemize}
    \item If compiling is finished successfully, without error code, code box goes green
  \end{itemize}
  \item Process(or I/O)
  \begin{enumerate}
    \item Waiting for finish of compiling
    \item Check whether error code exists or not (No error code = Success)
  \end{enumerate}
\end{enumerate}

\textit{}

\subsection{Auto Compile Test - Failure}

\textit{}
\begin{figure}[h]
\centering
\includegraphics[width=0.5\textwidth]{./figures/UI_code_validation_fail.jpg}
%text slot
\caption{Concept Image of Code Testing Failure}
\label{fig_concept_fail}
\end{figure}


\begin{enumerate}
  \item Description
  \begin{itemize}
    \item If there is error code after compiling, program shows the error code to user (see Figure~\ref{fig_concept_fail})
  \end{itemize}
  \item Process(or I/O)
  \begin{enumerate}
    \item Waiting for finish of compiling
    \item Show error code result to user
    \item Enable users to edit codes if they can find the error.
  \end{enumerate}
\end{enumerate}

\textit{}

\subsection{Auto Compile Test - Edit Code Snippet}
\begin{enumerate}
  \item Description
  \begin{itemize}
    \item When the auto compile test ended with error, users are able to edit code manually
  \end{itemize}
  \item Process(or I/O)
  \begin{enumerate}
    \item Input: Error signal from auto compile test
    \item Output: Enabling users to edit code
  \end{enumerate}
\end{enumerate}


\textit{}


\subsection{Cancellation(2) - Clicking Return Button}
\begin{enumerate}
  \item Description
  \begin{itemize}
    \item After finishing compiling or during compiling, user can go back with clicking return button
  \end{itemize}
  \item Process(or I/O)
  \begin{enumerate}
    \item Clicking return button
    \item Go back to the code selection window
  \end{enumerate}
\end{enumerate}
\textit{}


\subsection{Cancellation(2) - Cliking X Window Button}
\begin{enumerate}
  \item Description
  \begin{itemize}
    \item After finishing compiling or during compiling, user can exit program
  \end{itemize}
  \item Process(or I/O)
  \begin{enumerate}
    \item Clicking X window button
    \item Terminate program
  \end{enumerate}
\end{enumerate}
\textit{}


\subsection{Selection - Proceed without testing}
%put warning window
\begin{enumerate}
  \item Description
  \begin{itemize}
    \item Users are able to procced without auto compile test
    \item Autocompile tested bits off
  \end{itemize}
  \item Process(or I/O)
  \begin{enumerate}
    \item Users click next button
    \item Proceed to process 3
  \end{enumerate}
\end{enumerate}
\textit{}


\subsection{Prompt - Cancellation}
\begin{enumerate}
  \item Description
  \begin{itemize}
    \item When user requests cancellation, prompt window appears, reconfirming cancellation
  \end{itemize}
  \item Process(or I/O)
  \begin{enumerate}
    \item Clicking Return or X window button
    \item Reconfirming cancellation, Yes or No
  \end{enumerate}
\end{enumerate}
\textit{}



\subsection{Prompt - Proceed}
\begin{enumerate}
  \item Description
  \begin{itemize}
    \item Before proceeding compiling, prompt window appears, reconfirming proceeding it
  \end{itemize}
  \item Process(or I/O)
  \begin{enumerate}
    \item Clicking code
    \item Reconfirming proceeding, Yes, or No
  \end{enumerate}
\end{enumerate}
\textit{}


\textbf{Process 3. Manual Testing and Result Saving}


\textit{}
\begin{figure}[h]
\centering
\includegraphics[width=0.5\textwidth]{./figures/UI_code_validation_fail.jpg}  
%text slot
\caption{Concept Image of Code Validation}
\label{fig_concept_validation_manual}
\end{figure}


\subsection{Result - UI}
\begin{enumerate}
  \item Description
  \begin{itemize}
    \item Showing result and providing process check and copy function to user (see Figure~\ref{fig_concept_validation_manual}).
  \end{itemize}
  \item Process(or I/O)
  \begin{enumerate}
    \item 3 buttons – Input, Output, Copy Code
  \end{enumerate}
\end{enumerate}
\textit{}

\subsection{Process Checking - Input \& Output}
\begin{enumerate}
  \item Description
  \begin{itemize}
    \item User can check whether code is what they wanted or not through manual checking process
    \item Once user pushes input value into the program, they can see there is expected output or not.
  \end{itemize}
  \item Process(or I/O)
  \begin{enumerate}
    \item Input: Input value into program formed with selected code
    \item Output: Print output from pushed input
  \end{enumerate}
\end{enumerate}
\textit{}


\subsection{Copy Code}

\begin{enumerate}
  \item Description
  \begin{itemize}
    \item Providing copy the whole code which is selected, easily
  \end{itemize}
  \item Process(or I/O)
  \begin{enumerate}
    \item Clicking copy button
    \item Copy selected code on the clipboard
  \end{enumerate}
\end{enumerate}
\textit{}



\subsection{Cancellation(4) - Clicking Return Button}
\begin{enumerate}
  \item Description
  \begin{itemize}
    \item User can go back with clicking return button
  \end{itemize}
  \item Process(or I/O)
  \begin{enumerate}
    \item Clicking return button
    \item Go back to the code selection window
  \end{enumerate}
\end{enumerate}
\textit{}


\subsection{Cancellation(4) - Cliking X Window Button}
\begin{enumerate}
  \item Description
  \begin{itemize}
    \item User can exit program
  \end{itemize}
  \item Process(or I/O)
  \begin{enumerate}
    \item Clicking X window button
    \item Program closing
  \end{enumerate}
\end{enumerate}
\textit{}

\section{Architecture Design and Implementation} % (fold)
\label{sec:architecture_design_and_implementation}

\subsection{Overall architecture} % (fold)
\label{sub:overall_architecture}

see Figure~\ref{overall}

% subsection overall_architecture (end)

\subsection{Directory Organization} % (fold)
\label{sub:directory_organization}

\begin{table}[h]
\renewcommand{\arraystretch}{1.3}
\caption{Directory Organization}
\label{table:directory_org}
\centering
\begin{adjustbox}{width=0.5\textwidth}
\small
\begin{tabular}{c||c||c}
\hline
\bfseries Directory & \bfseries File Names & \bfseries Module Names \\
\hline\hline
project/src/  & \parbox[t]{5cm}{main.py \\ setup.py} & \parbox[t]{5cm}{building up}\\
\hline
project/src/keyword\_/  & \parbox[t]{5cm}{synonym\_test.csv \\ keyword\_search.py \\ topic\_analysis.py} & \parbox[t]{5cm}{keyword\_search}\\
\hline
project/src/crawling/ & \parbox[t]{5cm}{crawling\_common.py \\ crawling\_stack.py \\ crawling\_google.py} & \parbox[t]{5cm}{get\_code}\\
\hline
project/src/comp\_exec/ &\parbox[t]{5cm}{error\_argument\_C++.py \\ execution\_C++.py \\ error\_argument\_py.py \\ execution\_py.py \\ ... } & \parbox[t]{5cm}{validation}\\
\hline
project/src/GUI/ & \parbox[t]{5cm}{search.py \\candidates.py \\ compiling.py \\ error.py \\ success.py} & \parbox[t]{5cm}{GUI}\\
\hline
project/utill/  & \parbox[t]{5cm}{/cpp/... \\ /py/...} & \parbox[t]{5cm}{compiling and execution related}\\
\hline

\end{tabular}
\end{adjustbox}
\end{table}

% subsection directory_organization (end)
\textbf{Module Description}

\subsection{keyword - keyword\_search} % (fold)
\label{sub:keyword_search}
\begin{figure}[ht]
\centering
\includegraphics[width=0.5\textwidth]{./figures/keyword_search.png}
%description of module get\_info
\caption{Description of module \textit{keyword\_search}}
\label{keyword_search}
\end{figure}


This module is to take real language from user in sentence or words, split up, and replace them to keywords for search. Our program is to detect keywords from natural language, and make coded files for output. For that, a csv file for keywords is needed. If this list of keywords is too short, all it can make is an error sign. So, we first made 2000 words for key, and make it extend. What we need for search a word in some sentences is first, split function. By that, we can save our resources for replacement, reduce error cases, or decrease misunderstanding. Then it replaces natural language to limited number of keyword. Those keywords are sent to next stage. And it analyzes frequency, deviation, and tendency of keyword composition for later use. The synonym\_word.csv is for those keywords, but there will be needs of complement. So, with analysis of keywords, the csv file extends. 


see Figure~\ref{keyword_search}
% subsection keyword_search (end)

\subsection{get\_code - Crawling and Selecting} % (fold)
\label{sub:get_code}
\begin{figure}[ht]
\centering
\includegraphics[width=0.5\textwidth]{./figures/get_info.png}
%description of module get\_info
\caption{Description of module \textit{get\_info}}
\label{get_info}
\end{figure}

This module search for appropriate codes using keywords from the \textit{query} module.
\textit{get\_code} imports \textit{requests} for query requesting to the server, mostly using REST API supported from forums.
For raw html files, the module imports \textit{beautifulsoup4} package to parse the code.

Secondly, search for codes in Google and retrieve result.
To get proper result from Google, we spoofed request header to firefox 53.0
For codes from Github, The program manipulates the source to get raw file.

After crawling codes from sites, this module also evaluates the quality of the code.
Codes are evaluated in three ways.
Firtly, codes are evaluated with the common properties like lenght of the codes.
Secondly, codes are evaluated with code specific properties. 
For example, the program check number of imports when evaluating python code.
Lastly, codes are evaluated with the origin, where the code came from.
For example, codes are highly evaluated when connected with google.

see Figure~\ref{get_info}
% subsection search_crawling (end)

\subsection{validation - Compling and Execution} % (fold)
\label{sub:validation}

\label{sub:validation}
\begin{figure}[ht]
\centering
\includegraphics[width=0.5\textwidth]{./figures/comp_exec.png}
%description of module get\_info
\caption{Description of module \textit{validation}}
\label{validation}
\end{figure}

This module is to take cpp file generated after parsing codes, then compile, and execute it.
It imports \textit{subprocess}, which makes python freely handle other programming language.
It creates command with Popen function in subprocess, and then executes it with communicate function in subprocess.
It compiles test.cpp using g++ and executes it using "./test" command
Error handling is made by error exception method. 
At this point, it imports os, sys, and subprocess. 
Each of these catches error which they can handle.
The os uses errorno, strerror, and filename.
The sys uses exc\_info() function, and the subprocess uses returncode and strip() function.
Then Error\_Argument prints out error caused within os, sys, and process.

see Figure~\ref{validation}
% subsection validation (end)


\subsection{GUI} % (fold)
\label{sub:gui}
\begin{figure}[ht]
\centering
\includegraphics[width=0.3\textwidth]{./figures/gui_overall.png}
%description of module get\_info
\caption{Description of module \textit{gui}}
\label{gui}
\end{figure}
GUI consists of 5 windows corresponds to each procedure.
Firstly, the \textit{main} window provides search.
After clicking the search, the program executes \textit{keyword\_search} and then window moves to \textit{search\_result} window.
\textit{search\_result} window has candidates from \textit{get\_info} module.

Clicking \textit{compile} button on the  \textit{search\_result} window will execute compile.
Compile is done by \textit{validation} module. After compiling, the program shows either \textit{success} or \textit{fail} window


see Figure~\ref{gui}
% subsection gui (end)


% section architecture_design_and_implementation (end)



\section{Use Case} % (fold)
\label{sec:use_case}

\begin{figure}[ht]
\centering
\includegraphics[width=0.3\textwidth]{./figures/use_case_overall.jpg}
%text slot
\caption{Overall Picture of Usecase}
\label{fig:usecase}
\end{figure}

This section illustrates whole usecase. User will use Coding Helper in pretty linear way. Firstly, for the first time of meeting the program.
The user will guided to install and learn the program.
After installing and learning, users will be able to access search and validate to easily find the codes from questions.

Figure~\ref{fig:usecase}. shows overall usecase.  

\subsection{Installation and First Use}
\begin{table}[ht]
\renewcommand{\arraystretch}{1}
\caption{Installation and First Use}
\label{table:usecase1}
\centering
\begin{adjustbox}{width=0.3 \textwidth}
\small
\begin{tabular}{c|c}
\hline
\multicolumn{2}{c}{\textbf{Use Case Over View}} \\
\hline
\textbf{Name of Use Case} & Installation and Initiation \\
\hline
\textbf{Actor} & User \\
\hline
\textbf{Entry Condition} & First Download\\
\hline
\multicolumn{2}{c}{\textbf{Flow of Events}}\\
\hline
\multicolumn{2}{c}{
\parbox[t]{5cm}{
  1. install \\
  2. first run the program \\
  3. do the tutorial \\
  4. open configuration window \\
  5. select languages and confirm installation \\
  6. open readme \\
  7. run the program(main.exe) \\
  +. tooltips
  }
}\\
\hline

\end{tabular}
\end{adjustbox}
\end{table}
%

\subsubsection{Install}
This use case specifies the way user install the program

\begin{itemize}
  \item download the file from the link
  \item auto extractor will unzip the program
  \item displays the progress
  \item show pragram installed finished display
\end{itemize}
\textit{}



\subsubsection{First Run The Program}
This use case presents how the program will act when the program is run very first time
\begin{itemize}
  \item initial running the program
  \item show up the configuration window(conf.exe)
  \item after configuration done, do the tutorial
\end{itemize}
\textit{}



\subsubsection{Do The Tutorial}
This use case specifies how the program will act when user do the tutorial
\begin{itemize}
  \item clicking the tutorial button on main window or initial setting 
  \item play the tutorial movie(tut.mov)
  \item after playing the tutorial, close the tut.mov
\end{itemize}
\textit{}



\subsubsection{Open Configuration Window}
This use case specifies how the configuration window looks like when the user opens the configuration window
\begin{itemize}
  \item initial running the program or double clicking on conf.exe
  \item open a checkbox gui
  \item the checkbox gui has languages to be installed
\end{itemize}
\textit{}



\subsubsection{Select Languages And Confirm Installation}
This use case specifies how the program should act in the configuration window. In the configuration window, the user can easily install language packs from server.
\begin{itemize}
  \item check language packs to install / uncheck languages to uninstall
  \item click install button to install and uninstall selected ones
\end{itemize}
\textit{}



\subsubsection{Open Readme}
This use case describes about opening the readme file.
\begin{itemize}
  \item open a readme file which gives basic installation
\end{itemize}
\textit{}

\subsubsection{Run The Program(main.exe)}
This use case specifies of initializing the program when the user executes the program.
\begin{itemize}
  \item running the program
  \item open the main window(search window)
\end{itemize}
\textit{}


\subsubsection{Tooltips}
This use case specifies about how the user enables the tooltips
\begin{itemize}
  \item hanging mouse on the buttons for 3 seconds
  \item show the tooltips
  \item tooltip disappear 1 seconds after mouse hovering finished
\end{itemize}
\textit{}

\subsection{Search}

\begin{table}[hbt]
\renewcommand{\arraystretch}{1}
\caption{Search the query}
\label{table:usecase2}
\centering
\begin{adjustbox}{width=0.3\textwidth}
\small
\begin{tabular}{c|c}
\hline
\multicolumn{2}{c}{\textbf{Use Case Over View}} \\
\hline
\textbf{Name of Use Case} & Installation and Initiation \\
\hline
\textbf{Actor} & User \\
\hline
\textbf{Entry Condition} & Run the Program\\
\hline
\multicolumn{2}{c}{\textbf{Flow of Events}}\\
\hline
\multicolumn{2}{c}{
\parbox[t]{5cm}{
  1. view search history \\
  2. enter a query \\
  3. search auto completion \\
  4. submit the query
  }
}\\
\hline

\end{tabular}
\end{adjustbox}
\end{table}


\subsubsection{View Search History}
This use case specifies about viewing search history. User can view the search history and replicates the search result using search history.
\begin{itemize}
  \item clicking search history button
  \item show search history pane
\end{itemize}
\textit{}
This use case sepecfies how users can search their questions. The users first select the language and then search on window
\begin{itemize}
  \item enter a query on the search window
  \item while entering a query, keep run auto completion
  \item show the auto completion result
  \item select language to search : first set to C language
  \item uninstalled languages are greyed out
\end{itemize}
\textit{}


\subsubsection{Search Auto Completion}
This use case specifies how the program finds search from user input. The program suggests alternatives or synonyms for quality of the search.
\begin{itemize}
  \item click 'search' button
  \item run, \textit{keyword\_search} module
  \item the program look up the `synonym\_test.csv'
  \item keyword\_search module extracts keyword under the user input
\end{itemize}
\textit{}

\subsubsection{Submit The Query}
This use case specifies the procedures when submitting the query
\begin{itemize}
  \item the program gets keyword
  \item run, query extracting module
  \item run, \textit{get\_code} module based on the result
  \item get\_code parses the code
  \item show the candidates window
\end{itemize}
\textit{}

\subsection{Validation}

\begin{table}[hbt]
\renewcommand{\arraystretch}{1}
\caption{Validation}
\label{table:usecase3}
\centering
\begin{adjustbox}{width=0.3\textwidth}
\small
\begin{tabular}{c|c}
\hline
\multicolumn{2}{c}{\textbf{Use Case Over View}} \\
\hline
\textbf{Name of Use Case} & Validation \\
\hline
\textbf{Actor} & User \\
\hline
\textbf{Entry Condition} & Select Code\\
\hline
\multicolumn{2}{c}{\textbf{Flow of Events}}\\
\hline
\multicolumn{2}{c}{
\parbox[t]{5cm}{
  1. select code \\
  2. modify code \\
  3. compile code \\
  4. detect error \\
  5. Supplement code \\
  6. Pop up fail window \\
  7. Pop up success window \\
  8. Copy code
  }
}\\
\hline

\end{tabular}
\end{adjustbox}
\end{table}



\subsubsection{Select Code}
This use case specifies about how user can interact with \textit{candidates} window. This is the main interaction with the program.
\begin{itemize}
  \item view each code to compile
  \item the user evaluates codes
  \item click code that the user things the best suitable : first set to 1st radio button
\end{itemize}
\textit{}


\subsubsection{Modify Code}
This use case specifies about modifiying the code if the user wants to.
\begin{itemize}
  \item the user decide whether the codes needs to be refined
  \item click on the code
  \item editing the code
  \item press enter to confirm
\end{itemize}
\textit{}


\subsubsection{Compile Code}
This use case specifies about compiling the code, when user clicks copile button.
\begin{itemize}
  \item after selecting and modifying, user press compile button
  \item program makes program file(test.py/cpp) with code
  \item validation executes compiler with subprocess
  \item validation makes execution file(test.out) if compiling succeeded
  \item validation runs execution file(test.out)
  \item program prints out the output of code
  \item show the result of compiling either \textit{success} or \textit{fail}
\end{itemize}
\textit{}



\subsubsection{Detect Error}
This use case specifies the way program detect error and print it while compiling

\begin{itemize}
  \item validation executes compiler with subprocess   
  \item when error occured, validation gets error code
  \item validation generates error message from error code
  \item program pints out the error message
\end{itemize}
\textit{}



\subsubsection{Supplement Code}
This use case specifies the way program supplement code when detected error can be fixed by itself
\begin{itemize}
  \item when error occured, validation gets error code
  \item validation generates error message from error code
  \item if error message is related with absence of some lines(ex.header), validation supplements code
  \item validation compiles supplemented code
\end{itemize}
\textit{}



\subsubsection{Pop Up Fail Window}
This use case specifies when the compile fails, shows error result.
\begin{itemize}
  \item failure of the compiling triggers the fail window
  \item show the fail window
  \item fail window shows errors
\end{itemize}
\textit{}



\subsubsection{Pop Up Success Window}
This use case specifies when the compile succeed, shows copy code window.
\begin{itemize}
  \item pops up the window
  \item the success window provides to show the resulting cod
  \item the success window also provides copying code
\end{itemize}
\textit{}

\subsubsection{Copy Code}
This use case specifies when the user wants to save result, this provides the way to save the result.
\begin{itemize}
  \item press copy code button
  \item code will be copied to the clipboard
\end{itemize}
\textit{}


\subsection{Shutdown}

\begin{table}[hbt]
\renewcommand{\arraystretch}{1}
\caption{Shutdown}
\label{table:usecase4}
\centering
\begin{adjustbox}{width=0.3\textwidth}
\small
\begin{tabular}{c|c}
\hline
\multicolumn{2}{c}{\textbf{Use Case Over View}} \\
\hline
\textbf{Name of Use Case} & Shutdown\\
\hline
\textbf{Actor} & User \\
\hline
\textbf{Entry Condition} & \\
\hline
\multicolumn{2}{c}{\textbf{Flow of Events}}\\
\hline
\multicolumn{2}{c}{
\parbox[t]{5cm}{
  1. Tooltips \\
  2. Click X Button \\
  3. Program Exit
  }
}\\
\hline

\end{tabular}
\end{adjustbox}
\end{table}


\subsubsection{Click X Button}
This use case specifies how the program act when x buttons are clicked
\begin{itemize}
  \item click x button
  \item pops prompt window that asks about the cancellation
  \item cancel all procedure and go back to the first search window
\end{itemize}
\textit{}

\subsubsection{Program Abort}
This use case specifies how the program closes when user aborted the program.
\begin{itemize}
  \item program terminated with unexpected event
  \item program terminates with default cleanup process
\end{itemize}

\subsubsection{Program Exit}
This use case specifies how to close the program.
\begin{itemize}
  \item click x button on main(search) window
  \item program terminates
\end{itemize}
\textit{}



% section use_case (end)

\section{Installation Guide}%(fold)
\label{sec:installation guide}
\textit{}

\subsection{Install Guide}
  We pre-compiled all necessary files so user does not have to install and setting.
  Users just have to 1. download all the file, 2. extract file and 3. just execute it.

  \subsubsection{Download Files}
    We uploaded portable package at google drive and made direct download URL. User can simply download this portable package at 
    \url{https://drive.google.com/open?id=0B9aiOILFud23c1FXbkN6azM3WmM}.

    See Figure~\ref{fig_download}
      \begin{figure}[ht]
      \centering
      \includegraphics[width=0.3\textwidth]{./figures/downloadURL.jpg}
      %text slot
      \caption{Downloading the files.}
      \label{fig_download}
      \end{figure}

  \subsubsection{Extract Files}
    \begin{figure}[ht]
    \centering
    \includegraphics[width=0.3\textwidth]{./figures/extraction.jpg}
    %text slot
    \caption{Extraction of the files.}
    \label{fig_extraction}
    \end{figure}

    \begin{figure}[ht]
    \centering
    \includegraphics[width=0.3\textwidth]{./figures/extracting.jpg}
    %text slot
    \caption{Extracting the files.}
    \label{fig_extracting}
    \end{figure}
    Downloaded file name is exe.win-amd64-3.6.zip. Extract this file with own extraction program such as WinZip, WinRAR, 7-zip, Alzip, Bandizip, and so on.
    Extraction takes time, so wait with a cup of tea.

    See Figure~\ref{fig_extraction}. and Figure~\ref{fig_extracting}.



  \subsubsection{Excute the file} 
    \begin{figure}[ht]
    \centering
    \includegraphics[width=0.3\textwidth]{./figures/execution.jpg}
    %text slot
    \caption{execution of the files.}
    \label{fig_excution}
    \end{figure}
  In exe.win-amd64-3.6.zip, there is main.exe, which is one-click-executable file. User only need to double click it after extracting exe.win-amd64-3.6.zip. For convinience, we made a shortcut to execute main.exe on the root directory.
  See Figure~\ref{fig_excution}.
  \subsubsection{User Guide}
  User Guide is included with the name README.docx

  Or see Appendix~\ref{user_guide}.

\subsection{Included Modules}
\begin{enumerate}
\item{Python}
\begin{itemize}
\item Our program is based on python. For our program, we used Python3.6.0
User should download Python3.6.0 for executing our program
So we enclose Python36 in our installation package.
\end{itemize}
\item{PyQt5}
\begin{itemize}
\item UI of our program is made with PyQt5.
User should download PyQt5 for seeing User Interface of our program.
So we enclose PyQt5 in our installation package.
\end{itemize}
\item{Dev-C++}\\
\begin{itemize}
\item For compiling code, user should download GNU Compiler Collection.
So we enclose Dev-C++ in our installation package.
\end{itemize}
\end{enumerate}

%section installation guid end
\vfill

\onecolumn
\appendices


\section{Overall Architecture}
\begin{figure}[ht]
\centering
\includegraphics[width=0.9\textwidth]{./figures/overall_arch.jpg}
%description of module get\_info
\caption{Description of Overall Architecture}
\label{overall}
\end{figure}

\pagebreak

\section{User Guide}
\label{user_guide}
\begin{figure}[ht]
\centering
\includegraphics[width=0.8\textwidth]{./figures/1-3.png}
%description of module get\_info
\caption{}
\label{user1}
\end{figure}

\begin{figure}[ht]
\centering
\includegraphics[width=0.8\textwidth]{./figures/4-5.png}
%description of module get\_info
\caption{}
\label{user2}
\end{figure}

\begin{figure}[ht]
\centering
\includegraphics[width=0.8\textwidth]{./figures/6.png}
%description of module get\_info
\caption{}
\label{user3}
\end{figure}

\begin{figure}[ht]
\centering
\includegraphics[width=0.8\textwidth]{./figures/7-8.png}
%description of module get\_info
\caption{}
\label{user4}
\end{figure}

% ''-------------end of the document-------------------''
% % figure example --------------
% \begin{figure}[h]
% \centering
% \includegraphics[width=0.5\textwidth]{./figures/UI_main.jpg}
% %text slot
% \caption{Example Example.}
% \label{fig_sim}
% \end{figure}

% section specifications (end)
% \begin{thebibliography}{1}

% \bibitem{IEEEhowto:kopka}
% H.~Kopka and P.~W. Daly, \emph{A Guide to \LaTeX}, 3rd~ed.\hskip 1em plus
%   0.5em minus 0.4em\relax Harlow, England: Addison-Wesley, 1999.

% \end{thebibliography}




% that's all folks
\end{document}


